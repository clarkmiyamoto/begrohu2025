\section{Overview}
In these lectures, we are interested in creating phase diagrams, where the macrostate is wether the neural network learns the data distribution, as we vary relevant quantities of the network (i.e. size dependence, architecture, etc.)\\
\\
We'll go over
\begin{enumerate}
	\item Intro to Hebbian Networks
	\item Hopfield Model
	\item Restricted Boltzmann Machines and Hebbian Networks
	\item Architectures
\end{enumerate}

\section{A Model for Neurons}
Our story begins with a MC Oulloch (in 1943) trying to create a mathematical model for neurons. Neurons can be effectively described as nodes on a network
\begin{itemize}
	\item Edges of the graph physical correspond to "axon" (which are paths which connect neurons).
	\item They fire electrical signals which in-turn will choose wether or not another neuron will fire.
\end{itemize}
Now let's mathematize this. A system of $\{x_i: x_i \in \{0,1\}_{i=1}^N$ neurons. They are coupled with strengths $\{J_i : J_i \in \mathbb R\}_{i=1}^N$ to an output neuron $y \in \{0,1\}$. The state equation of the system is
\begin{align}
	y = \Theta(U - U^*), ~~~ \text{s.t. }U = \sum_i J_i x_i
\end{align}
where $\Theta$ is the heaviside function, and $U^* \in \mathbb R$ is interpreted as the activation energy for the neuron to fire.\\
\\
However, neurons don't just point to one "god" neuron $y$, they all point towards each other (albeit sparsely). So our true state equation should be
\begin{align}
	x_i = \Theta (U_i - U_i^*),  ~~~ \text{s.t. }U_i =\sum_{j=1}^N J_{ij} x_j
\end{align}
Ok, this is all cool, but what if we want to generlaize our model a little more? (Specifically we'd like to recover a spin-glass system that way we can take advantage of our knowledge from statistical physics). So a couple things
\begin{itemize}
	\item Consider defining a random variable $z_i \sim \sigma_z$ where $\mathbb E[z] =0$ and $\mathbb E[z] =1$.
	\item Instead of using binary variables $\underline x = (x_1, ..., x_N) \in \{0,1\}^N$, what if we could use ising-spin variables $\underline \sigma \in (\sigma_1, ..., \sigma_N) \in \{\pm 1\}^N$. We can transform between the two using	\begin{align}
		x_i = \frac{1}{2} (1 + \sigma_i)
	\end{align}
\end{itemize}
Using this transformaiton, we can rewrite our state equation in-terms of $\underline \sigma$
\begin{align}
	\sigma_i^{(t+1)} = \text{sgn}[\varphi_k^{(t)} + T z_i^{(t)}], ~~~ \text{s.t. } \varphi_i= \sum_j J_{ij}\sigma_i^{(t)} + h
\end{align}


\begin{definition}
	The random variable $X \sim \text{Rad}(p)$ is defined to have the following properties
	\begin{align}
		\mathbb P(x=1) = 1 - \mathbb P(x = -1) = \frac{1+p}{2}
	\end{align}
\end{definition}
\begin{hypothesis}
	We assume the probability distribution over $p_Z$ is symmetric. That is $p_Z(-z) = p_Z(z)$
\end{hypothesis}
If we assume this hypothesis to hold, and define the following
\begin{align}
	g_Z(z) \equiv \int_{-\infty}^z p_Z(u) du = \mathbb P[Z < z]
\end{align}
We can show that probability of $\mathbb P [\sigma_i^{(t+1)} = \pm 1] = g_Z(\pm \frac{\varphi_i ^{(t)}}{T})$.
\\
\\
Also some notation for the future, we'll notate $p_Z(z) \equiv \frac{1}{2}[1 -\tanh^2(z)]$ and therefore $g_Z(z) =  \frac{1}{2}[1 + \tanh(z)]$.

\begin{definition}
	[Update / Dynamics] 
	We define two ways to perform dynamics
	\begin{itemize}
		\item Sequential. We select $i$ randomly, and update $\mathbb P[\sigma_i^{(t+1)}] = g(\sigma_i \varphi_i^{(t)} / T) $
		\item Parallel. $\mathbb P[\sigma^{(t+1)}] = \prod_i g(\sigma)i \varphi_i^{(t)} / T)$
	\end{itemize}
\end{definition},
Finally, we we assume the coupling matrix is symmetry $(J^T = J)$, $J_{ii} \geq 0$ and $h$ is stationary. Then the Lyaponov function
\begin{align}
	L = - \frac{1}{2} \underline \sigma^T J \underline \sigma - \underline h^T \underline\sigma
\end{align}
What's nice about this function is that is always is lowered with the dynamics we've asserted
\begin{align}
	\Delta L (\sigma, J) \leq 0
\end{align}
where $\Delta L$ is meant to be interpreted as a discrete difference in time $L(t=t'+\Delta t') - L(t=t')$. You can also show that $L$ is lower bounded.\\
\\
The result of all of this, if you run your update step, you can create a diagram which shows where you end up at fix points dependent on where you start. So if data is encoded in $\bar \sigma$, and data flows to fixed points (attractors) $\underline \xi$ depending on initialization--- we have created a classifier!
\begin{definition}
	[Attractor] We define the attractor $\xi \in \{\pm 1\}^N$, is the configuration of spins $\underline \sigma$ where you flow to a fixed point. So $\underline \xi \in \{\{ \{\pm 1\}^N\}^M$ is defines the set of attractors emitted by the model.
\end{definition}

 However, to make a \emph{good} classifier, we'd like to be able to tune/train our couplings $J$  s.t.
\begin{itemize}
	\item We can host a many attractors / stationary points
	\item The radius between the initalization and the attract is big enough s.t. we can find attractors. Aka generalize. 
\end{itemize}
\begin{definition}
	[Retrivial] A definition for when the network emits attractors
	\begin{itemize}
		\item (Strong Definition). $J = J(\xi)$ retries $\underline \xi$ staritng from $\underline \sigma$. 
		\item (Weak Definition). $\underline \xi $ is $(\delta,\epsilon)$ stable. That is when your state enter a $\delta$-radius ball away from the attractor, it will forever stay within an $\epsilon$-radius ball to the attractor.
\end{itemize}
\end{definition}
Ok, to make our life easier, assume we know what we want our attractors to be $\{\xi^\mu\}_{\mu =1}^K$, and all we need to do is tune our interaction matrix $J$. We can use \textbf{Hebb's Rule} to learn the parameters.
\begin{definition}
	[Hebb's Rule] Assert $J_{ij}  = \frac{1}{N}\sum_{\mu=1}^K \xi_i^\mu \xi_j^\mu$. The $1/N$ is to make the rule non-extensive in the number of spins $N$. We perform the update
	\begin{align}
		J^{(t+1)}_{ij} \leftarrow J_{ij}^{(t)} + \frac{1}{N} \xi_i \xi_j
	\end{align}
	We note that the time scale of the couplings $J$ is drastically different from the time scale of the neurons $\underline x$.
\end{definition}

\begin{sidework}
	In the case the number of attractors is $K=1$.
	\begin{align}
		L(\underline \sigma) & = \frac{1}{2N}\sum_{ij} \sigma_i \sigma_j \xi_i \xi_j\\
		& = - \frac{1}{2N} \sum_{ij} \tilde \sigma_i \tilde \sigma_j \geq - \frac{N}{2}
	\end{align}
	where the transformation we did was $\sigma_i \to \tilde \sigma_i \xi_i$. So we've recovered a Potts Model!
\end{sidework}
\begin{definition}
	[Motts Magnetization] We define an overlap parameter
	\begin{align}
		m_\mu \equiv \frac{1}{N} \sum_i \tilde \sigma_i = \frac{1}{N} \sum_i \sigma_i \xi^\mu_i \in [-1, +1]
	\end{align}
	What is nice about this parameter is that it allows us to access the quality of retrivial. That is if your states $\underline \sigma$ actually go to the attractor $\underline \xi$, then $m_1 \to +1$. 
\end{definition}
\begin{theorem}
	[Kohonen's Projector Rule]
	If $J$ is s.t. $\sum_j J_{ij} \xi_j^\mu = \lambda \xi_i^\mu$ for $\lambda >0$ for all $i$. Then
	\begin{align}
		\text{sgn}(J \underline \xi^\mu)=  \text{sgn}( \lambda \underline \xi^\mu) = \underline \xi^\mu
	\end{align}
	This means that $\underline \xi$ is a fixed point of the dynamics.
\end{theorem}
\subsection{Noiseless Dynamics}
Consider the following system
\begin{align}
	\xi^1 \varphi_i (\xi^1) > 0
\end{align}
If we expand this quantity this
\begin{align}
	\xi_i^1 \varphi_i(\xi^1) = \frac{\xi_i^1}{N} \sum_{j=1, j \neq i}^N \sum_\mu^K \xi_i^\mu \xi_j^\mu \xi_j^1 = \frac{N-1}{N} + \frac{1}{N} \sum_{\mu > 1} \sum_{j=1, j\neq i} \xi_i^1 \xi_i^\mu \xi_j^\mu \xi_j^1
\end{align}
We can see this leads to a signal-to-noise decomposition...


\section{Jun 9th}
Let $\underline \sigma = \{\pm 1\}^{N}$ bet a set of neurons, $\underline \xi^\mu \in \{\pm 1\}^N$ be the set of memories $\mu = 1,... , K$, and $J = \frac{1}{N} \xi \cdot \xi^T$ and $J_{ij} = \frac{1}{N} \sum_\mu \xi_i^\mu \xi_j^\mu$, with the update step
\begin{align}
	\underline\sigma^{(t+1)} = \text{sgn}(\underline \varphi^{(t)})\\
	\varphi_i^{(t)} \equiv \varphi_i(\underline \sigma^{(t)}) = \sum_j J_{ij} \sigma_j^{(t)}
\end{align}
The question we'll be answering today is $\underline \sigma = \underline \xi^\mu$ a stable solution? For simplicity let's assume $K=1$, so $\mu=1$ is our target configuraton
\begin{align}
	\xi_i^1 = \sgn(\sum_k J_{ik} \xi_k^1)
\end{align}

\begin{align}
	\xi_i^1 \sum_{j \neq i} J_{ij} \xi_{j}^1 = \underbrace{\frac{N-1}{N}}_{\mu=1, \text{ Term S}} + \frac{1}{N}\underbrace{\sum_{j \neq i} \sum_{\mu > 1, \text{ Term R}} \xi_i^1 \xi_j^1 \xi_i^\mu \xi_j^\mu}_{\mu > 1} \label{eqn:Jun9th_ExpandedTerm}
\end{align}
Note the second term is $\sqrt{(K-1)(N-1)}/N \approx \sqrt{K/N}$. So as long as $K/N \to_{N \to \infty} 0$, then the first term will be greater than the absolute value fo the second term.
\begin{sidework}
	Consider an initial configuration of the neurons $\underline \sigma^{(0)} = \underline \xi^\mu \odot \underline \chi^\mu$ where $\chi_i^\mu \sim \text{Rad}(p)$ (meaning $\mathbb P[\chi_i^\mu = \pm 1] = (1 + p)/2$) and $\odot$ is the Hammard product (element wise multiplication). This is supposed to be interpreted as a $\chi$ perturbation away from the memory $\underline \xi^\mu$.
\end{sidework}
\begin{sidework}
Consider $\underline \sigma^{(0)} = \sgn(\underline \xi^1 + \underline \xi^2 + \underline \xi^3)$. You can show it is also stable under $K/N \to 0$.\\
\\
In General $\underline \sigma^{(0)} = \sgn(\sum_{\mu=1}^L\underline\xi^\mu )$ where $L$ is odd is stable under $K/N \to 0$.
\end{sidework}
So we've seen that $K/N$ is crucial to determining stability in the system. So let's give it a name

\begin{definition} [Load] We define the \textbf{load} quantity
	\begin{align}
		\alpha \equiv \lim_{n \to \infty} \frac{K}{N}
	\end{align}When 
	\begin{itemize}
		\item $\alpha = 0$, is \textbf{low load}. 
		\item $\alpha > 0$ is \textbf{high-load}
	\end{itemize}
\end{definition}

\subsection{Hopfield Model} 
Consider the Hamiltonian
\begin{align}
	H_N(\underline \sigma; J) = - \frac{1}{N} \sum_{i < j} J_{ij} \sigma_i \sigma_j - \sum_i h_i \sigma_i\\
	\mathcal H (\underline \sigma; J) = - \frac{1}{N} \sum_{i < j} \sum_\mu \xi_i^\mu \xi_j^\mu \sigma_i \sigma_j - \sum_\mu \sum_i \xi_i^\mu h_i \sigma_i
\end{align}
where $\xi_i^\mu \sim \text{Rad}(0)$. We also introduce some notation $\omega(\dot) = \sum_{\underline \sigma} (\cdot) P_N(\underline \sigma, \underline \xi)$, $\mathbb E[\cdot] = \sum_{\underline \xi}(\cdot)$, and $\langle \cdot \rangle _N = \mathbb E \omega_N(\cdot)$. We also denote hte Mattis Magnetization as $m_\mu \equiv \frac{1}{N} \sum_i \xi_i^\mu \sigma_i \in [-1, +1]$.

So now going back to $\mathcal H$, we can rewrite it int this notation as
\begin{align}
	\mathcal H_N & = - \frac{1}{2N} \sum_{\mu} N m_\mu N m_\mu + \frac{1}{2} \sum_{\tau, \mu} (1) - \sum_\mu m_\mu N h_\mu\\
	& = - \frac{N}{2} \underline m^2 + \frac{K}{2} - N \underline h^T \underline m
\end{align}
Now our goal is to compute $\langle m \rangle$. We can accomplish this by computing the free energy $f_N(\xi) = \frac{1}{N\beta} \log Z_n(\xi)$ and use it as the moment generating function.






























